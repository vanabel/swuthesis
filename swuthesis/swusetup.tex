% !TeX root = ./main.tex

\swusetup{
  title              = {西南大学学位论文模板},
  subtitle	     = {---示例文档 v\swuthesisversion},
  title*             = {An example of thesis template for Southwest University v\swuthesisversion},
  author             = {李泽平},
  author*            = {Li Zeping},
  speciality         = {数学与应用数学(师范)},
  speciality*        = {Mathematics and Applied Mathematics},
  supervisor         = {XXX~教授, XXX~教授},
  supervisor*        = {Prof. XXX, Prof. XXX},
  % date               = {2017-05-01},  % 默认为今日
  % professional-type  = {专业学位类型},
  % professional-type* = {Professional degree type},
  department         = {数学与统计学院},  % 院系,本科生需要填写
  department*        = {School of Mathematics and Statistics, Southwest University},  
  student-id         = {222019314000000},  % 学号,本科生需要填写
  address            = {重庆 400715},
  address*           = {Chongqing 400715},
  % secret-level       = {秘密},     % 绝密|机密|秘密|控阅,注释本行则公开
  % secret-level*      = {Secret},  % Top secret | Highly secret | Secret
  % secret-year        = {10},      % 保密/控阅期限
  % reviewer           = true,      % 声明页显示“评审专家签名”
  %
  % 数学字体
  % math-style         = GB,  % 可选:GB, TeX, ISO
  math-font          = xits,  % 可选:stix, xits, libertinus
}


% 加载宏包

% 定理类环境宏包
\usepackage{amsthm}

% 插图
\usepackage{graphicx}

% 三线表
\usepackage{booktabs}

% 跨页表格
\usepackage{longtable}

% 算法
\usepackage[ruled,linesnumbered]{algorithm2e}

% SI 量和单位
\usepackage{siunitx}

% 参考文献使用 BibTeX + natbib 宏包
% 顺序编码制
\usepackage[sort]{natbib}
% 本科生要求的著录体例:切换为 swuthesis-bachelor
\bibliographystyle{\bibpath swuthesis-bachelor}

% 著者-出版年制
% \usepackage{natbib}
% \bibliographystyle{swuthesis-authoryear}

% 本科生参考文献的著录格式
% \usepackage[sort]{natbib}
% \bibliographystyle{swuthesis-bachelor}

% 参考文献使用 BibLaTeX 宏包
% \usepackage[style=swuthesis-numeric]{biblatex}
% \usepackage[bibstyle=swuthesis-numeric,citestyle=swuthesis-inline]{biblatex}
% \usepackage[style=swuthesis-authoryear]{biblatex}
% \usepackage[style=swuthesis-bachelor]{biblatex}
% 声明 BibLaTeX 的数据库
% \addbibresource{bib/swu.bib}

% 配置图片的默认目录
\graphicspath{{figures/}}

% 数学命令
\makeatletter
\newcommand\dif{%  % 微分符号
  \mathop{}\!%
  \ifswu@math@style@TeX
    d%
  \else
    \mathrm{d}%
  \fi
}
% 参考文献载入
\providecommand*{\bibpath}{}
\renewcommand*{\bibpath}{./vendor/}
\makeatother
\newcommand\eu{{\symup{e}}}
\newcommand\iu{{\symup{i}}}

% 用于写文档的命令
\DeclareRobustCommand\cs[1]{\texttt{\char`\\#1}}
\DeclareRobustCommand\env[1]{\texttt{#1}}
\DeclareRobustCommand\pkg[1]{\textsf{#1}}
\DeclareRobustCommand\file[1]{\nolinkurl{#1}}

% hyperref 宏包在最后调用
\usepackage{hyperref}

% —— 本科生定制覆盖(满足学院“详解”要求)——
% 1) 页边距:A4 纸,四边 2.5cm
% 2) 取消页眉,仅保留页脚居中页码(从目录开始连续编阿拉伯数字)
% 3) 正文(mainmatter)采用 1.5 倍行距
\makeatletter
% 四边 2.5cm 覆盖类里的 geometry 设置
\AtBeginDocument{\geometry{margin=2.5cm}}

% 统一页脚页码居中,小四(12bp 近似),无页眉
\fancypagestyle{plain}{%
  \fancyhf{}%
  \fancyfoot[C]{\fontsize{12bp}{12bp}\selectfont\thepage}%
  \renewcommand{\headrulewidth}{0pt}%
  \renewcommand{\footrulewidth}{0pt}%
}
\fancypagestyle{headings}{%
  \fancyhf{}%
  \fancyfoot[C]{\fontsize{12bp}{12bp}\selectfont\thepage}%
  \renewcommand{\headrulewidth}{0pt}%
  \renewcommand{\footrulewidth}{0pt}%
}
\pagestyle{plain}
\makeatother
